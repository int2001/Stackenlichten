\documentclass[border=0mm]{standalone}
\usepackage[no-math]{fontspec}
\usepackage{xunicode,xltxtra}
\usepackage{tikz,pgfplots}
\usetikzlibrary{calc,backgrounds,pgfplots.colormaps}
% \defaultfontfeatures{Mapping=tex-text}
% \setromanfont[Ligatures={Common},ItalicFont={Vollkorn-Italic},BoldFont={Vollkorn-Bold}, BoldItalicFont={Vollkorn-BoldItalic}]{Vollkorn}
%\setromanfont[Ligatures={Common}]{Gill Sans}
%\setromanfont[Ligatures={Common}]{Futura}
\setromanfont[Ligatures={Common}]{Futura Black}
%
%\definecolor{bg}{rgb}{0,0,0}
%\definecolor{fg}{rgb}{1,1,1}
\definecolor{bg}{rgb}{1,1,1}
\definecolor{fg}{rgb}{0,0,0}
%\definecolor{hl}{HTML}{1A202C}
% 129793 - 505050 - FFF5C3 - 9BD7D5 - FF7260
\definecolor{hl}{HTML}{FF7260} % left reddish
\definecolor{hl2}{HTML}{FFF5C3} % right yellowish
\definecolor{hl3}{HTML}{98D7D5} % top blueish
\tikzset{base/.style={hl}}
\tikzset{main/.style={fill=bg,draw=fg,thick}}
\tikzset{mydots/.style={line width=3pt, line cap=round, dash pattern=on 0pt off 1.5\pgflinewidth}}
\tikzset{mydotsL/.style={line width=5pt, line cap=round, dash pattern=on 0pt off 1.8085\pgflinewidth}}
\tikzset{verts/.style={hl!50!black!50!white,line width=1.25pt}}
\newcommand\wTh{0.4} %woodThickness, short W in comments
\newcommand\neoOff{.07} %neoclassizism
\begin{document}
  \begin{tikzpicture}[background rectangle/.style={fill=bg,draw=none},show background rectangle]
    % fix background of triangle
    \draw[fill=white,draw=bg] (-4,0) -- (4,0) -- (0,{4*sqrt(3)}) -- cycle;
    % draw circles
    % border
    \coordinate (A1) at (-4,0);
    \coordinate (A2) at (-2.5,0);
    \coordinate (A3) at (-2.5,\wTh);
    \coordinate (A4) at (-1.5,\wTh);
    \coordinate (A5) at (-1.5,0);
    \coordinate (A6) at (-.5,0);
    \coordinate (A7) at (-.5,\wTh);
    \coordinate (A8) at (.5,\wTh);
    \coordinate (A9) at (.5,0);
    \coordinate (A10) at (1.5,0);
    \coordinate (A11) at (1.5,\wTh);
    \coordinate (A12) at (2.5,\wTh);
    \coordinate (A13) at (2.5,0);
    \coordinate (A14) at (4,0);
    \path[shift={(-2,{2*sqrt(3)})},rotate=60,main] (-4,0) coordinate (B1)  (-2.5,0) coordinate (B2) (-2.5,-\wTh)
    coordinate (B3) (-1.5,-\wTh) coordinate (B4) (-1.5,0) coordinate (B5)
    (-.5,0) coordinate (B6) (-.5,-\wTh) coordinate (B7) (.5,-\wTh)
    coordinate (B8) (.5,0) coordinate (B9) (1.5,0) coordinate (B10)
    (1.5,-\wTh) coordinate (B11) (2.5,-\wTh) coordinate (B12) (2.5,0)
    coordinate (B13) (4,0) coordinate (B14);
    %
    \path[shift={(2,{2*sqrt(3)})},rotate=-60,main] (-4,0) coordinate (C1)  (-2.5,0) coordinate (C2) (-2.5,-\wTh)
    coordinate (C3) (-1.5,-\wTh) coordinate (C4) (-1.5,0) coordinate (C5)
    (-.5,0) coordinate (C6) (-.5,-\wTh) coordinate (C7) (.5,-\wTh)
    coordinate (C8) (.5,0) coordinate (C9) (1.5,0) coordinate (C10)
    (1.5,-\wTh) coordinate (C11) (2.5,-\wTh) coordinate (C12) (2.5,0)
    coordinate (C13) (4,0) coordinate (C14);

    % background
    \foreach \x in {0,0.05,...,1.0} {%
      \draw[draw=hl,draw opacity=0.5,thin,fill=hl,fill opacity=.1] ($(A1)!\x!(A14)$) -- ($(B14)!\x!(B1)$) -- (A1) -- cycle;
      \draw[draw=hl2,draw opacity=0.5,fill=hl2, fill opacity=.1] ($(A1)!\x!(A14)$) -- ($(C14)!\x!(C1)$) -- (A14) -- cycle;
      \draw[draw=hl3,draw opacity=0.5,fill=hl3, fill opacity=.1] ($(C14)!\x!(C1)$) -- ($(B14)!\x!(B1)$) -- (C1) -- cycle; 
    }
    \foreach \i in {1,2,3}{%
    \draw[black,thick]
    ($(A1)+(.55*\i*\neoOff,\i*\neoOff)$)
%    (A1)
    -- ($(A2)+(-\i*\neoOff,\i*\neoOff)$)
    -- ($(A3)+(-\i*\neoOff,\i*\neoOff)$) -- ($(A4) + (+\i*\neoOff,\i*\neoOff)$)
    -- ($(A5)+(+\i*\neoOff,\i*\neoOff)$) -- ($(A6) + (-\i*\neoOff,\i*\neoOff)$)
    -- ($(A7)+(-\i*\neoOff,\i*\neoOff)$) -- ($(A8) + (+\i*\neoOff,\i*\neoOff)$)
    -- ($(A9)+(+\i*\neoOff,\i*\neoOff)$) -- ($(A10) + (-\i*\neoOff,\i*\neoOff)$)
    -- ($(A11)+(-\i*\neoOff,\i*\neoOff)$) -- ($(A12) + (+\i*\neoOff,\i*\neoOff)$)
    -- ($(A13)+(+\i*\neoOff,\i*\neoOff)$)
    --
   ($(A14) + (-.55*\i*\neoOff,\i*\neoOff)$)
%   (A14)
    ;
    \draw[black,thick]
   ($(B1)+([rotate=60].55*\i*\neoOff,-\i*\neoOff)$)
%   (B1)
    -- ($(B2)+([rotate=60]-\i*\neoOff,-\i*\neoOff)$)
    -- ($(B3)+([rotate=60]-\i*\neoOff,-\i*\neoOff)$)
    -- ($(B4)+([rotate=60]+\i*\neoOff,-\i*\neoOff)$)
    -- ($(B5)+([rotate=60]+\i*\neoOff,-\i*\neoOff)$)
    -- ($(B6)+([rotate=60]-\i*\neoOff,-\i*\neoOff)$)
    -- ($(B7)+([rotate=60]-\i*\neoOff,-\i*\neoOff)$)
    -- ($(B8)+([rotate=60]+\i*\neoOff,-\i*\neoOff)$)
    -- ($(B9)+([rotate=60]+\i*\neoOff,-\i*\neoOff)$)
    -- ($(B10)+([rotate=60]-\i*\neoOff,-\i*\neoOff)$)
    -- ($(B11)+([rotate=60]-\i*\neoOff,-\i*\neoOff)$)
    -- ($(B12)+([rotate=60]\i*\neoOff,-\i*\neoOff)$)
    -- ($(B13)+([rotate=60]\i*\neoOff,-\i*\neoOff)$)
    --
   ($(B14)+([rotate=60]-.55*\i*\neoOff,-\i*\neoOff)$)
%    (B14)
    ;
    \draw[black,thick]
    ($(C1)+([rotate=-60].55*\i*\neoOff,-\i*\neoOff)$)
%    (C1)
    -- ($(C2)+([rotate=-60]-\i*\neoOff,-\i*\neoOff)$)
    -- ($(C3)+([rotate=-60]-\i*\neoOff,-\i*\neoOff)$)
    -- ($(C4)+([rotate=-60]+\i*\neoOff,-\i*\neoOff)$)
    -- ($(C5)+([rotate=-60]+\i*\neoOff,-\i*\neoOff)$)
    -- ($(C6)+([rotate=-60]-\i*\neoOff,-\i*\neoOff)$)
    -- ($(C7)+([rotate=-60]-\i*\neoOff,-\i*\neoOff)$)
    -- ($(C8)+([rotate=-60]+\i*\neoOff,-\i*\neoOff)$)
    -- ($(C9)+([rotate=-60]+\i*\neoOff,-\i*\neoOff)$)
    -- ($(C10)+([rotate=-60]-\i*\neoOff,-\i*\neoOff)$)
    -- ($(C11)+([rotate=-60]-\i*\neoOff,-\i*\neoOff)$)
    -- ($(C12)+([rotate=-60]\i*\neoOff,-\i*\neoOff)$)
    -- ($(C13)+([rotate=-60]\i*\neoOff,-\i*\neoOff)$)
    -- 
    ($(C14)+([rotate=-60]-.55*\i*\neoOff,-\i*\neoOff)$)
%   (C14)
    ;
    }
    \draw[bg,thick] (A1) -- (A14) -- (B14) -- cycle;
    \draw[fill=bg,draw=fg] (A2) -- (A3) -- (A4) -- (A5);
    \draw[fill=bg,draw=fg] (A6) -- (A7) -- (A8) -- (A9);
    \draw[fill=bg,draw=fg] (A10) -- (A11) -- (A12) -- (A13);
    \foreach \x[count=\xnext from 2] in {1,...,13}{
      \draw[main] (A\x) -- (A\xnext);
    }
    \draw[fill=bg,draw=fg] (B2) -- (B3) -- (B4) -- (B5);
    \draw[fill=bg,draw=fg] (B6) -- (B7) -- (B8) -- (B9);
    \draw[fill=bg,draw=fg] (B10) -- (B11) -- (B12) -- (B13);
    \foreach \x[count=\xnext from 2] in {1,...,13}{
      \draw[main] (B\x) -- (B\xnext);
    }
    \draw[fill=bg,draw=fg] (C2) -- (C3) -- (C4) -- (C5);
    \draw[fill=bg,draw=fg] (C6) -- (C7) -- (C8) -- (C9);
    \draw[fill=bg,draw=fg] (C10) -- (C11) -- (C12) -- (C13);
    \foreach \x[count=\xnext from 2] in {1,...,13}{
      \draw[main] (C\x) -- (C\xnext);
    }
    %
    \node (A) at (0,-.66) {\Huge\color{fg} Stackenlichten};
  \end{tikzpicture}
\end{document}
