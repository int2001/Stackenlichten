\documentclass[border=0mm]{standalone}
\usepackage[no-math]{fontspec}
\usepackage{xunicode,xltxtra}
\usepackage{tikz}
\usetikzlibrary{calc,backgrounds}
\defaultfontfeatures{Mapping=tex-text}
\setromanfont[Ligatures={Common},ItalicFont={Vollkorn-Italic},BoldFont={Vollkorn-Bold}, BoldItalicFont={Vollkorn-BoldItalic}]{Vollkorn}
%
% \definecolor{bg}{rgb}{0,0,0}
% \definecolor{fg}{rgb}{1,1,1}
% \colorlet{hl}{cyan}
\definecolor{bg}{rgb}{1,1,1}
%\definecolor{fg}{RGB}{68,48,34}
\definecolor{fg}{rgb}{1,1,1}
\colorlet{hl}{cyan!75!black}
\tikzset{base/.style={hl}}
\tikzset{main/.style={fill=bg}}
\newcommand\wTh{0.4} %woodThickness, short W in comments
\begin{document}
  \begin{tikzpicture}[background rectangle/.style={fill=bg,draw=none},show background rectangle]
    % fix background of triangle
    \draw[fill=fg,draw=bg] (-4,0) -- (4,0) -- (0,{4*sqrt(3)}) -- cycle;
    % draw circles
    \draw[fill=hl!25!fg,draw=hl!75!bg] (0,{4/3*sqrt(3)}) circle
      ({4/3*sqrt(3)-\wTh});
    \draw[fill=hl!50!fg,draw=hl!75!bg] (0,{4/3*sqrt(3)}) circle
       ({3/4*(4/3*sqrt(3)-\wTh)});
    \draw[fill=hl!75!fg,draw=hl!75!bg] (0,{4/3*sqrt(3)}) circle
       ({1/2*(4/3*sqrt(3)-\wTh)});
    \draw[fill=hl,draw=hl!75!bg] (0,{4/3*sqrt(3)}) circle
       ({1/4*(4/3*sqrt(3)-\wTh)});
    \draw[base] ({- ( 4- \wTh*cot(60) )}, \wTh) -- ({4- \wTh*cot(60)},\wTh);
    % border
    \draw[main] (-4,0) -- (-2.5,0) -- (-2.5,\wTh) -- (-1.5,\wTh) -- (-1.5,0) --
      (-.5,0) -- (-.5,\wTh) -- (.5,\wTh) -- (.5,0) -- (1.5,0) -- (1.5,\wTh) -- 
      (2.5,\wTh) -- (2.5,0) -- (4,0);
    %
    \begin{scope}[shift={(-2,{2*sqrt(3)})},rotate=60]
    \draw[base] ({- ( 4- \wTh*cot(60) )},-\wTh) -- ({4- \wTh*cot(60)},-\wTh);
    \draw[main] (-4,0) -- (-2.5,0) -- (-2.5,-\wTh) -- (-1.5,-\wTh) -- (-1.5,0) --
      (-.5,0) -- (-.5,-\wTh) -- (.5,-\wTh) -- (.5,0) -- (1.5,0) -- (1.5,-\wTh) -- 
      (2.5,-\wTh) -- (2.5,0) -- (4,0);
    \end{scope}
    \begin{scope}[shift={(2,{2*sqrt(3)})},rotate=-60]
    \draw[base] ({- ( 4- \wTh*cot(60) )},-\wTh) -- ({4- \wTh*cot(60)},-\wTh);
    \draw[main] (-4,0) -- (-2.5,0) -- (-2.5,-\wTh) -- (-1.5,-\wTh) -- (-1.5,0) --
      (-.5,0) -- (-.5,-\wTh) -- (.5,-\wTh) -- (.5,0) -- (1.5,0) -- (1.5,-\wTh) -- 
      (2.5,-\wTh) -- (2.5,0) -- (4,0);
    \end{scope}
    \node (A) at (0,-.66) {\Huge\color{hl} Stackenlichten};
  \end{tikzpicture}
\end{document}
